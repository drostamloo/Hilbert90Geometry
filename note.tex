\documentclass[letterpaper, openany, 10pt]{article}
\usepackage[headheight=20pt, total={5in, 8in}]{geometry}
%\usepackage{mathpazo}		% this is the 'proper' way to load palladio, but it breaks blackboard fonts in comparison to Vakil's text 
\usepackage{palatino}
\linespread{1.05}         % Palladio needs more leading (space between lines)
\usepackage[T1]{fontenc}
\usepackage{eulervm}	% Math font that goes well with palladio
%\usepackage{tgpagella} % Alternative font
\usepackage[svgnames]{xcolor}

\title{Hilbert 90 Geometry: Remarks on Birationality}
\author{}
\date{\today}

% All other preamble settings
%%% PACKAGES AND SETTINGS %%%

% math packages
\usepackage{mathtools}  % Includes amsmath
\usepackage{amssymb}    % Math symbols such as \mathbb
\usepackage{amsthm}	   % theorem environments
\usepackage{xpatch}    % Patch amsthm \swappedhead to add period after theorem head
\xpatchcmd\swappedhead{~}{.~}{}{}   % Add period after swapped-number theorem head
\usepackage{stmaryrd}   % more symbols
\usepackage{mathrsfs}   % RSFS font in math mode 
\usepackage{extarrows}	% extra extensible arrows beyond those in mathtools/amsmath
\usepackage{etoolbox}	% possibly for starting a new page after every solution

% More extensible arrows
\makeatletter
\providecommand*{\twoheadrightarrowfill@}{%
	\arrowfill@\relbar\relbar\twoheadrightarrow
}
\providecommand*{\twoheadleftarrowfill@}{%
	\arrowfill@\twoheadleftarrow\relbar\relbar
}
\providecommand*{\xtwoheadrightarrow}[2][]{%
	\ext@arrow 0579\twoheadrightarrowfill@{#1}{#2}%
}
\providecommand*{\xtwoheadleftarrow}[2][]{%
	\ext@arrow 5097\twoheadleftarrowfill@{#1}{#2}%
}
\makeatother

% typesetting sheaf Hom
\usepackage{urwchancal} % \mathpzc (alternate mathcal): also lowercase.
\DeclareFontFamily{OT1}{pzc}{}
\DeclareFontShape{OT1}{pzc}{m}{it}{<-> s * [1.10] pzcmi7t}{}
\DeclareMathAlphabet{\mathpzc}{OT1}{pzc}{m}{it}
\usepackage{calligra}	% for alternative typesetting of sheaf Hom

% other packages
\usepackage{pgffor} % Used for looping through subfiles, but note that loops will break inverse search
\usepackage{tikz-cd}
\usepackage{quiver}
\usepackage{graphicx}
\graphicspath{ {../assets/} }
\usepackage{enumitem}
\usepackage{hyperref}
\hypersetup{
	pdfauthor={Daniel Rostamloo},
    colorlinks=true,
    linktoc=page,     %set to all if you want both sections and subsections linked
	linkcolor=DarkBlue,	 %the value "." sets the link color to match the surrounding text color
	citecolor=DarkBlue,
	urlcolor=DarkBlue,
	bookmarksnumbered,
}

% PDF Bookmark Settings
\usepackage{bookmark}
\bookmarksetup{numbered}

\makeatletter
\bookmarksetup{%
	addtohook={%
		\ifnum\toclevel@part=\bookmarkget{level}\relax
			\renewcommand*{\numberline}[1]{Part #1. }%
		\fi
	},
}
\makeatother

\makeatletter
\bookmarksetup{%
	addtohook={%
		\ifnum\toclevel@chapter=\bookmarkget{level}\relax
			\renewcommand*{\numberline}[1]{Chapter #1. }%
		\fi
	},
}
\makeatother

\makeatletter
\bookmarksetup{%
	addtohook={%
		\ifnum\toclevel@section=\bookmarkget{level}\relax
			\renewcommand*{\numberline}[1]{#1. }%
		\fi
	},
}
\makeatother

% Citation/referencing packages
\usepackage[alphabetic]{amsrefs}
\usepackage[capitalize]{cleveref}

\crefname{exercise}{Exercise}{Exercises}
\Crefname{exercise}{Exercise}{Exercises}
\crefname{genericexercise}{Exercise}{Exercises}
\Crefname{genericexercise}{Exercise}{Exercises}
\crefname{theorem}{Theorem}{Theorems}
\Crefname{theorem}{Theorem}{Theorems}
\crefname{generictheorem}{Theorem}{Theorems}
\Crefname{generictheorem}{Theorem}{Theorems}
\crefname{lemma}{Lemma}{Lemmas}
\Crefname{lemma}{Lemma}{Lemmas}
\crefname{proposition}{Proposition}{Propositions}
\Crefname{proposition}{Proposition}{Propositions}
\crefname{corollary}{Corollary}{Corollaries}
\Crefname{corollary}{Corollary}{Corollaries}
\crefname{fact}{Fact}{Facts}
\Crefname{fact}{Fact}{Facts}
\crefname{claim}{Claim}{Claims}
\Crefname{claim}{Claim}{Claims}
\crefname{genericdiscussion}{\S}{\S}
\Crefname{genericdiscussion}{\S}{\S}
\crefname{definition}{Definition}{Definitions}
\Crefname{definition}{Definition}{Definitions}
\crefname{remark}{Remark}{Remarks}
\Crefname{remark}{Remark}{Remarks}

% proper inline math display, adjust height for symbols like \sum
%\everymath{\displaystyle}

%%% THEOREM ENVIRONMENTS %%%

\renewcommand\qedsymbol{$\blacksquare$}

\newtheoremstyle{statement}%			% Name
	{}%                                 % Space above
	{}%                                 % Space below
	{\itshape}%                         % Body font
	{}%                                 % Indent amount
	{\bfseries}%                        % Theorem head font
	{: --- }%                           % Punctuation after theorem head
	{10pt}%                                % Space after theorem head, ' ', or \newline
	{}%									% Custom head spec

\newtheoremstyle{definition}%			% Name
	{}%                                 % Space above
	{}%                                 % Space below
	{\mdseries}%                        % Body font
	{}%                                 % Indent amount
	{\itshape}%                         % Theorem head font
	{.}%                                % Punctuation after theorem head
	{10pt}%                                % Space after theorem head, ' ', or \newline
	{}%									% Custom head spec

\newtheoremstyle{discussion}%			% Name
	{}%									% Space above
	{}%									% Space below
	{\mdseries}%						% Body font
	{}%									% Indent amount
	{\bfseries}%						% Theorem head font
	{.}%								% Punctuation after theorem head
	{10pt}%								% Space after theorem head, ' ', or \newline
	{}%									% Custom head spec

\newtheoremstyle{exercise}%             % Name
	{}%                                 % Space above
	{}%                                 % Space below
	{\mdseries}%                        % Body font
  	{}%                                 % Indent amount
  	{}%                            		% Theorem head font
  	{.}%                                % Punctuation after theorem head
  	{10pt}%                             % Space after theorem head, ' ', or \newline
  	{\textbf{\thmnumber{#2}.} \textsc{\thmname{#1}}\thmnote{ (#3)}}%		% Custom head spec

% Exercises

\theoremstyle{exercise}
\newtheorem{exercise}{Exercise}[section]

\theoremstyle{exercise}
\newcommand{\thisexercisename}{}
\newtheorem{genericexercise}[exercise]{\thisexercisename}
\newenvironment{namedexercise}[1]
	{\renewcommand{\thisexercisename}{#1}%
	\begin{genericexercise}}
	{\end{genericexercise}}


\renewcommand*{\theexercise}{\thesection.\Alph{exercise}} % Redefine exercise counter to use letters instead of numbers

% Gives \begin{solution} same formating as \begin{proof}

\newenvironment{solution}
  {\begin{proof}[Solution]}
{\end{proof}\clearpage}

%\AtEndCommand{qedsymbol}{%
%	\clearpage%
%}

% Generic Statements

\theoremstyle{statement}
\swapnumbers 
% This is done to make the theorem counter appear first for the theorem environments created below. We cannot use the \swapnumbers command for the exercise environments above because the theorem head for the exercise environment uses multiple font styles which are best defined manually in the custom head spec.

\newtheorem{theorem}{Theorem}[section]
\newtheorem{lemma}[theorem]{Lemma}
\newtheorem{proposition}[theorem]{Proposition}
\newtheorem{corollary}[theorem]{Corollary}
\newtheorem{fact}[theorem]{Fact}
\newtheorem{claim}[theorem]{Claim}

% Discussions and Statements with Custom Names (Discussions are the designated format for examples)

\newcommand{\thistheoremname}{}
\newtheorem{generictheorem}[theorem]{\thistheoremname}
\newenvironment{namedtheorem}[1]
	{\renewcommand{\thistheoremname}{#1}%
	\begin{generictheorem}}
	{\end{generictheorem}}

\theoremstyle{discussion}
\newcommand{\thisdiscussionname}{}
\newtheorem{genericdiscussion}[theorem]{\thisdiscussionname}
\newenvironment{nameddiscussion}[1]
	{\renewcommand{\thisdiscussionname}{#1}%
	\begin{genericdiscussion}}
	{\end{genericdiscussion}}

% Definitions and Remarks

\theoremstyle{definition}
\newtheorem{definition}[theorem]{Definition}
\newtheorem{remark}[theorem]{Remark}

%%% CUSTOM MATH OPERATORS %%%

% Algebraic Geometry

\DeclareMathOperator{\Spec}{Spec}
\DeclareMathOperator{\Proj}{Proj}
\DeclareMathOperator{\Pic}{Pic}
\DeclareMathOperator{\Div}{div}
\DeclareMathOperator{\CaDiv}{CaDiv}
\DeclareMathOperator{\Cl}{Cl}
\DeclareMathOperator{\CaCl}{CaCl}
\DeclareMathOperator{\HOM}{\mathpzc{Hom}}
\DeclareMathOperator{\EXT}{\mathpzc{Ext}}
\DeclareMathOperator{\Supp}{Supp}
\DeclareMathOperator{\Bl}{Bl}

% alternative sheaf Hom and sheaf Ext typesetting
\DeclareMathOperator{\SheafHom}{\mathscr{H}\kern -3pt \text{{\calligra\large om}}\,}
\DeclareMathOperator{\SheafExt}{\mathscr{E}\text{\kern -3pt {\calligra\large xt}}\,}

% Other Algebra

\DeclareMathOperator{\pr}{pr}
\DeclareMathOperator{\nil}{nil}
\DeclareMathOperator{\Hom}{Hom}
\DeclareMathOperator{\codim}{codim}
\DeclareMathOperator{\Aut}{Aut}
\DeclareMathOperator{\End}{End}
\DeclareMathOperator{\colim}{colim}
\DeclareMathOperator{\characteristic}{char}
\DeclareMathOperator{\id}{id}
\DeclareMathOperator{\Span}{Span}
\DeclareMathOperator{\sgn}{sgn}
\DeclareMathOperator{\Tr}{Tr}
\DeclareMathOperator{\N}{N}
\DeclareMathOperator{\im}{im}
\DeclareMathOperator{\coim}{coim}
\DeclareMathOperator{\coker}{coker}
\DeclareMathOperator{\HH}{H}
\DeclareMathOperator{\hh}{h}
\DeclareMathOperator{\rank}{rank}
\DeclareMathOperator{\acts}{\curvearrowright}
\DeclareMathOperator{\trdeg}{tr. deg}
\DeclareMathOperator{\Tor}{Tor}
\DeclareMathOperator{\Ext}{Ext}
\DeclareMathOperator{\Gal}{Gal}
\DeclareMathOperator{\sep}{sep}
\DeclareMathOperator{\Syz}{Syz}
\DeclareMathOperator{\pd}{pd}
\DeclareMathOperator{\depth}{depth}
\DeclareMathOperator{\bm}{bm}
\DeclareMathOperator{\burch}{burch}
\DeclareMathOperator{\length}{length}
\DeclareMathOperator{\socle}{socle}
\DeclareMathOperator{\Char}{char}
\DeclareMathOperator{\Sym}{Sym}
\DeclareMathOperator{\Ann}{Ann}
\DeclareMathOperator{\Ass}{Ass}


\begin{document}

\maketitle

\section{Remarks on Birationality}

Let $L/k$ be a cyclic field extension of degree $n$, with $p \coloneqq \characteristic k$. Let $G \coloneqq \Gal(L/k) = \left\langle \sigma \right\rangle$. Recall the maps
\begin{gather*}
	f: L^\times \to L^x \\
	x \mapsto x/\sigma(x)
\end{gather*}
and
\begin{gather*}
	h: L \times L \to L \\
	(y, z_0) \mapsto \sigma^0(z_0) + y \cdot \sigma(y) \cdot \sigma^2(z_0) + \ldots + y \cdot \sigma(y) \cdot \ldots \cdot \sigma(y)^{n-2} \cdot \sigma^{n-1}(z_0)
\end{gather*}

We know $f$ maps surjectively onto the subset $X \coloneqq \{y \in L \mid N(y) = 1\}$. From Miles' note, we know that $z_0 \in L$ can be fixed so that $\Tr(z_0) \neq 0$ in order for $h(1, z_0) \neq 0$, which guarantees that $h_{z_0} \coloneqq h|_{L \times \{z_0\}}$, thought of as a polynomial endomorphism in the coefficients of $L \cong k^n$, is at least nonzero at $1 \in L$. If we instead regard $h: L \to \Hom_k(L, L)$ as a function of $y$, we can also recall from the note that
\begin{gather*}
	h \circ f: L^\times \to \Hom_k(L, L) \\
	x \mapsto (z_0 \mapsto x \cdot \Tr(\frac{z_0}{x})).
\end{gather*}

Let $1, x_1, \ldots, x_{n-1}$ be a $k$-basis for $L$, and let
\[
	x = a_0 + a_1x_1 + \ldots + x_{n-1}
\]
be an element of $L$ for some coefficients $a_i \in K$. Let $H$ be the hypersurface in $\mathbb{P}_k^n = \Proj k[a_0, \ldots, a_{n-1}, y]$ given by the equation $N(x) = y^n$. By an abuse of notation, let $f$ denote the map $\mathbb{P}_k^{n-1} = \Proj k[a_0, \ldots, a_{n-1}] \to H$ induced by $f$ as referenced above, and let $g$ denote the map $H \to \mathbb{P}_k^{n-1}$ induced by $h_{z_0}$.

In this note, we will build up to the following proposition:

\begin{proposition}
	If $L/k$ as above is either an Artin-Schreier or Kummer extension, then $f$ is a birational morphism with inverse $g$.
\end{proposition}

We first establish some basic facts. Let $x_0, \ldots, x_{n-1}$ denote a $k$-basis for $L$ such that we represent an arbitrary element $x \in L$ by the sum
\[
	x = \sum_{i=0}^{n-1} a_i x_i
\]
for some indeterminates $a_0, \ldots, a_{n-1}$ representing a choice of coefficients in $k$. With this notation, the field norm $N(x) \coloneqq N_{L/k}(x)$ can be regarded as a polynomial in the indeterminates $a_0, \ldots a_{n-1}$. We recall the following theorem:

\begin{theorem}[\cite{flanders}]
	Let $K = k(\theta)$ be a primitive extension of $k$. Then the general norm $N(x)$ (as a polynomial function of the $a_i$) is irreducible in $k[x]$.
\end{theorem}

\begin{lemma}
	The polynomial $1 - N(x)$ is irreducible.
\end{lemma}

\begin{proof}
	By \cite[Theorem 1]{flanders} above, we know that the polynomial $N(x)$ in the $a_i$ is irreducible. We first show that $1-N(x)$ is irreducible over $k$. To show that $1 - N(x)$ is irreducible, it suffices to show that its homogenization in $\mathbb{P}_k^n$, namely $y^n - N(x)$, is irreducible. By \cite[Exercise 4.5.F(c)]{vakil}, one can check the primeness of a homogeneous ideal or element by considering its divisibility by homogeneous elements.

	So, suppose that $y^n - N(x) = pq$ for two homogeneous polynomials $p,q \in k[a_0, \ldots, a_{n-1}, y]$. We can regard each of $p,q$ as polynomials in $y$ with coefficients in $k[a_0, \ldots, a_{n-1}]$. With this view, it is clear that the product of the terms of $p$ and $q$ which do not involve $y$ is just $N(x)$. But by the irreducibility of $N(x)$, this means that one of $p$ or $q$ must be $1$. Then our homogeneity assumption forces one of $p$ or $q$ to be an element of $k$ and therefore a unit, and this finishes.

	To see that $y^n - N(x)$ is geometrically irreducible, we suppose again that there is some $p,q \in \overline{k}[a_0, \ldots, a_{n-1}, y]$ for which $y^n - N(x) = pq$. Let $l/k$ denote the finite field extension generated by the coefficients of $p$ and $q$ over $k$. Then $y^n - N(x)$ is reducible over $l$. But $N(x)$ describes the field norm of the degree $n$ extension of $l$, so this reducibility contradicts the theorem above.
\end{proof}

\begin{corollary}
	The hypersurface $X$ is integral.
\end{corollary}

\begin{lemma}
	The maps $f$ and $g$ are rational. If $k$ is infinite, then $g$ is also dominant.
\end{lemma}

\begin{proof}
	Both $f$ and $g$ are clearly rational -- they are regular over the complement of the simultaneous vanishing loci of their respective coordinates. If $k$ is infinite, then the subset of $k$-points of $\mathbb{P}_k^{n-1}$ is dense. Let $P \subset \mathbb{P}_k^{n-1}$ denote the subset of $k$-points. The computations from Miles' note show that $(g \circ f)|_P = \id_P$. Hence, $g$ surjects onto the $k$-points of $\mathbb{P}_k^{n-1}$, so the image of $g$ is dense in $\mathbb{P}_k^{n-1}$, and this finishes.
\end{proof}

Since $g$ is dominant, we can realize the function field $K(X)$ as an extension of $K(\mathbb{P}_k^{n-1}) = k(t_1, \ldots, t_{n-1})$. By the principal ideal theorem, we also know that $\dim X = n-1$, so $K(X)$ is also a field of transcendence degree $n-1$. It is then a straightforward algebraic exercise to show that the extension $K(X)/K(\mathbb{P}_k^{n-1})$ must be algebraic. 

\begin{claim}
	There is a dense open subset of $\mathbb{P}_k^{n-1}$ over which the following identity holds:
	\[
		g \circ f = \id_{\mathbb{P}_k^{n-1}}.
	\]
\end{claim}

\begin{proof}[Sketch of idea]
	It suffices to show this identity after base-change to the algebraic closure, so we may suppose that $k = \overline{k}$. Since $k$ can be taken to be infinite in this way, the subset of $k$-points, which are precisely the closed points, of $\mathbb{P}_k^{n-1}$ is dense. It would be nice to write some formula for the maps which would give a bijection on these $\overline{k}$ points. If one can show that $X$ is normal, then it might be possible to use Zariski's Main Theorem (e.g.\cite[\href{https://stacks.math.columbia.edu/tag/03GW}{Lemma 03GW}]{stacks-project} and \cite{264216}) to argue that $f$ and $g$ must be birational inverses. 

	Another possibility is working within $k \neq \overline{k}$. Without modifying the existing coordinates, it might be possible to show that the $g \circ f$ and $\id_{\mathbb{P}_k^{n-1}}$ induce the same endomorphism of the $\overline{k}$-points $\mathbb{P}_k^{n-1}(\overline{k})$ to apply the fact referenced \href{https://math.stackexchange.com/q/630465}{here}. One could then immediately apply this to the field extension discussed above to see that $K(X) \cong K(\mathbb{P}_k^{n-1})$, and this would finish.

	Applying these ideas to Artin-Schreier extensions of finite fields would then be a matter of applying descent: since $\SheafHom$ is a sheaf in any Grothendieck topology, we can change base by the faithfully flat map $\Spec k(t) \to \Spec k$ to apply the ideas above which require an infinite base field.
\end{proof}

% Bibliography

\bibliography{note}

\end{document}
